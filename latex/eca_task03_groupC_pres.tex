\pdfminorversion=4
\documentclass{beamer}

%\usepackage[utf8]{inputenc}
\usepackage[english]{babel}
\usepackage{graphicx}
\usepackage{beamerthemedefault}
\usepackage{bbding} %used for checkmarks in itemize

\setbeamertemplate{navigation symbols}{}
\setbeamertemplate{footline}[frame number] 

\newcommand*\tick{\item[\Checkmark]}
\newcommand*\fail{\item[\XSolidBrush]}
\newcommand*\hand{\item[\HandRight]}

\title{ECA group 12 \\ Task 2}
\author{\small{P\"ollitsch, Pr\"unster ,Sommer ,Vierthaler}}
\date{May 2013}


\begin{document}
%-----------------------------------------------------------------------
\begin{frame}[plain]
	\titlepage
\end{frame}

%-----------------------------------------------------------------------
\begin{frame}
\frametitle{summary of our task02}
order type creation of a given pointset\\
\begin{itemize}
\item focus on combinatoric attributes
\item using tools presented in the lecture
\item grid size and amount of points limited
\end{itemize}
thoughts about \\
\begin{itemize}
\item collinear points within $\lambda$-matrix
\item automatic / semiautomatic creation of point set from $\lambda$-matrix
\item creation of all $\lambda$-matrices
\end{itemize}

\end{frame}

%-----------------------------------------------------------------------
\begin{frame}
\frametitle{our practical goals}
java program
\begin{itemize}
\tick gives user visible input/output
\tick calculates order types for different labellings (left handed orientation)
\tick chooses minimal order type (=fingerprint)
\hand semiautomatic tool for creating point sets from $\lambda$
\fail count all order types for given amount of points
\end{itemize}
\end{frame}
%-----------------------------------------------------------------------
\begin{frame}
\frametitle{screenshot: init, mark collinear points}
\includegraphics[width=\textwidth]{figures/00_program_init_4_points.png}
\end{frame}
%-----------------------------------------------------------------------
%\begin{frame}
%\frametitle{screenshot: delete points}
%\includegraphics[width=\textwidth]{figures/01_program_delete.png}
%\end{frame}
%-----------------------------------------------------------------------
\begin{frame}
\frametitle{screenshot: correct labeling according to fingerprint}
\includegraphics[width=\textwidth]{figures/02_program_labeling.png}
\end{frame}
%-----------------------------------------------------------------------
%\begin{frame}
%\frametitle{screenshot: move points}
%\includegraphics[width=\textwidth]{figures/04_program_move_points.png}
%\end{frame}
%-----------------------------------------------------------------------
%\begin{frame}
%\frametitle{screenshot: select single point}
%\includegraphics[width=\textwidth]{figures/05_program_select_points.png}
%\end{frame}
%-----------------------------------------------------------------------
%\begin{frame}
%\frametitle{screenshot: select multiple points}
%\includegraphics[width=\textwidth]{figures/06_program_multiple_selection.png}
%\end{frame}
%-----------------------------------------------------------------------
\begin{frame}
\frametitle{screenshot: create points from $\lambda$ - matrix: init}
\includegraphics[width=\textwidth]{figures/07_point_creation.png}
\end{frame}
%-----------------------------------------------------------------------
\begin{frame}
\frametitle{screenshot: create points from $\lambda$ - matrix: step 1}
\includegraphics[width=\textwidth]{figures/07_point_creation_step1.png}
\end{frame}
%-----------------------------------------------------------------------
\begin{frame}
\frametitle{screenshot: create points from $\lambda$ - matrix: step 2}
\includegraphics[width=\textwidth]{figures/07_point_creation_step2.png}
\end{frame}
%-----------------------------------------------------------------------
\begin{frame}
\frametitle{screenshot: create points from $\lambda$ - matrix: valid}
\includegraphics[width=\textwidth]{figures/07_point_creation_valid.png}
\end{frame}

%-----------------------------------------------------------------------
%\begin{frame}
%\section{recap $\lambda$-matrix}
%\frametitle{$\lambda$-matrix creation}
%\begin{tabular}{| c || c | c | c | c |}
%    & 1 & 2 & 3 & 4 \\
% \hline
%  1 & - & \textbf{0} & 1 & 2 \\
%  2 & 2 & - & 0 & 1 \\
%  3 & 1 & 2 & - & 0 \\
%  4 & 0 & 1 & 2 & - \\
%\end{tabular}
%\begin{minipage}{0.5\textwidth}
%\includegraphics[width=200px]{figures/lambda_not_colinear_12}
%\end{minipage}
%\end{frame}
%
%%-----------------------------------------------------------------------
%\begin{frame}
%\frametitle{$\lambda$-matrix creation}
%\begin{tabular}{| c || c | c | c | c |}
%    & 1 & 2 & 3 & 4 \\
% \hline
%  1 & - & 0 & \textbf{1} & 2 \\
%  2 & 2 & - & 0 & 1 \\
%  3 & 1 & 2 & - & 0 \\
%  4 & 0 & 1 & 2 & - \\
%\end{tabular}
%\begin{minipage}{0.5\textwidth}
%\includegraphics[width=200px]{figures/lambda_not_colinear_13}
%\end{minipage}
%\end{frame}
%
%%-----------------------------------------------------------------------
%\begin{frame}
%\frametitle{$\lambda$-matrix creation}
%\begin{tabular}{| c || c | c | c | c |}
%    & 1 & 2 & 3 & 4 \\
% \hline
%  1 & - & 0 & 1 & \textbf{2} \\
%  2 & 2 & - & 0 & 1 \\
%  3 & 1 & 2 & - & 0 \\
%  4 & 0 & 1 & 2 & - \\
%\end{tabular}
%\begin{minipage}{0.5\textwidth}
%\includegraphics[width=200px]{figures/lambda_not_colinear_14}
%\end{minipage}
%\end{frame}
%
%%-----------------------------------------------------------------------
%\begin{frame}
%\frametitle{$\lambda$-matrix creation}
%\begin{tabular}{| c || c | c | c | c |}
%    & 1 & 2 & 3 & 4 \\
% \hline
%  1 & - & 0 & 1 & 2 \\
%  2 & \textbf{2} & - & 0 & 1 \\
%  3 & 1 & 2 & - & 0 \\
%  4 & 0 & 1 & 2 & - \\
%\end{tabular}
%\begin{minipage}{0.5\textwidth}
%\includegraphics[width=200px]{figures/lambda_not_colinear_21}
%\end{minipage}
%\end{frame}

%-----------------------------------------------------------------------
% % hide following pictures
% \begin{frame}
% \frametitle{$\lambda$-matrix creation}
% \begin{tabular}{| c || c | c | c | c |}
%     & 1 & 2 & 3 & 4 \\
%  \hline
%   1 & - & 0 & 1 & 2 \\
%   2 & 2 & - & \textbf{0} & 1 \\
%   3 & 1 & 2 & - & 0 \\
%   4 & 0 & 1 & 2 & - \\
% \end{tabular}
% \begin{minipage}{0.5\textwidth}
% \includegraphics[width=200px]{figures/lambda_not_colinear_23}
% \end{minipage}
% \end{frame}
% 
% %-----------------------------------------------------------------------
% \begin{frame}
% \frametitle{$\lambda$-matrix creation}
% \begin{tabular}{| c || c | c | c | c |}
%     & 1 & 2 & 3 & 4 \\
%  \hline
%   1 & - & 0 & 1 & 2 \\
%   2 & 2 & - & 0 & \textbf{1} \\
%   3 & 1 & 2 & - & 0 \\
%   4 & 0 & 1 & 2 & - \\
% \end{tabular}
% \begin{minipage}{0.5\textwidth}
% \includegraphics[width=200px]{figures/lambda_not_colinear_24}
% \end{minipage}
% \end{frame}
% 
% %-----------------------------------------------------------------------
% \begin{frame}
% \frametitle{$\lambda$-matrix creation}
% \begin{tabular}{| c || c | c | c | c |}
%     & 1 & 2 & 3 & 4 \\
%  \hline
%   1 & - & 0 & 1 & 2 \\
%   2 & 2 & - & 0 & 1 \\
%   3 & \textbf{1} & 2 & - & 0 \\
%   4 & 0 & 1 & 2 & - \\
% \end{tabular}
% \begin{minipage}{0.5\textwidth}
% \includegraphics[width=200px]{figures/lambda_not_colinear_31}
% \end{minipage}
% \end{frame}
% 
% %-----------------------------------------------------------------------
% \begin{frame}
% \frametitle{$\lambda$-matrix creation}
% \begin{tabular}{| c || c | c | c | c |}
%     & 1 & 2 & 3 & 4 \\
%  \hline
%   1 & - & 0 & 1 & 2 \\
%   2 & 2 & - & 0 & 1 \\
%   3 & 1 & 2 & - & 0 \\
%   4 & \textbf{0} & 1 & 2 & - \\
% \end{tabular}
% \begin{minipage}{0.5\textwidth}
% \includegraphics[width=200px]{figures/lambda_not_colinear_41}
% \end{minipage}
% \end{frame}
% 
% %-----------------------------------------------------------------------
% \begin{frame}
% \frametitle{$\lambda$-matrix creation}
% \begin{tabular}{| c || c | c | c | c |}
%     & 1 & 2 & 3 & 4 \\
%  \hline
%   1 & - & 0 & 1 & 2 \\
%   2 & 2 & - & 0 & 1 \\
%   3 & 1 & 2 & - & 0 \\
%   4 & 0 & \textbf{1} & 2 & - \\
% \end{tabular}
% \begin{minipage}{0.5\textwidth}
% \includegraphics[width=200px]{figures/lambda_not_colinear_42}
% \end{minipage}
% \end{frame}
% 
% 
% %-----------------------------------------------------------------------
% \begin{frame}
% \frametitle{$\lambda$-matrix creation}
% \begin{tabular}{| c || c | c | c | c |}
%     & 1 & 2 & 3 & 4 \\
%  \hline
%   1 & - & 0 & 1 & 2 \\
%   2 & 2 & - & 0 & 1 \\
%   3 & 1 & 2 & - & 0 \\
%   4 & 0 & 1 & \textbf{2} & - \\
% \end{tabular}
% \begin{minipage}{0.5\textwidth}
% \includegraphics[width=200px]{figures/lambda_not_colinear_43}
% \end{minipage}
% \end{frame}
% 
% %-----------------------------------------------------------------------
 \begin{frame}
 \frametitle{$\lambda$-matrix creation}
 \begin{tabular}{| c || c | c | c | c |}
     & 1 & 2 & 3 & 4 \\
  \hline
   1 & - & 0 & 1 & 2 \\
   2 & 2 & - & 0 & 1 \\
   3 & 1 & 2 & - & 0 \\
   4 & 0 & 1 & 2 & - \\
 \end{tabular}
 \begin{minipage}{0.5\textwidth}
 \includegraphics[width=200px]{figures/lambda_not_colinear}
 \end{minipage}
 \end{frame}
% 
%-----------------------------------------------------------------------
%-----------------------------------------------------------------------
\begin{frame}

\frametitle{$\lambda$-matrix symmetry}
\begin{center}
\begin{tabular}{| c || c | c | c | c |}
    & 1 & 2 & 3 & 4 \\
 \hline
  1 & - & \textcolor{red}{0} & \textcolor{blue}{1} & \textcolor{violet}{2} \\
  2 & \textcolor{red}{2} & - & \textcolor{green}{0} & \textcolor{orange}{1} \\
  3 & \textcolor{blue}{1} & \textcolor{green}{2} & - & \textcolor{magenta}{0} \\
  4 & \textcolor{violet}{0} & \textcolor{orange}{1} & \textcolor{magenta}{2} & - \\
\end{tabular}
\end{center}
\vspace{3em}
sum is always n-2 ... in this case: 2
\end{frame}
% 
%-----------------------------------------------------------------------
\begin{frame}
\section{$\lambda$-matrix for collinear point sets}
\frametitle{collinear points}

lets look at collinear points
\end{frame}
%-----------------------------------------------------------------------

%-----------------------------------------------------------------------
\begin{frame}
\frametitle{$\lambda$-matrix creation for collinear sets}
\begin{tabular}{| c || c | c | c | c |}
    & 1 & 2 & 3 & 4 \\
 \hline
  1 & - & \textbf{0} & \textbf{0} & 2 \\
  2 & 1 & - & \textbf{0} & 1 \\
  3 & 1 & 1 & - & 0 \\
  4 & 0 & 1 & 2 & - \\
\end{tabular}
\begin{minipage}{0.5\textwidth}
\includegraphics[width=200px]{figures/lambda_colinear_12_13}
\end{minipage}
\end{frame}

%-----------------------------------------------------------------------
\begin{frame}
\frametitle{$\lambda$-matrix creation for collinear sets}
\begin{tabular}{| c || c | c | c | c |}
    & 1 & 2 & 3 & 4 \\
 \hline
  1 & - & 0 & 0 & \textbf{2} \\
  2 & 1 & - & 0 & 1 \\
  3 & 1 & 1 & - & 0 \\
  4 & 0 & 1 & 2 & - \\
\end{tabular}
\begin{minipage}{0.5\textwidth}
\includegraphics[width=200px]{figures/lambda_colinear_14}
\end{minipage}
\end{frame}

%-----------------------------------------------------------------------
\begin{frame}
\frametitle{$\lambda$-matrix creation for collinear sets}
\begin{tabular}{| c || c | c | c | c |}
    & 1 & 2 & 3 & 4 \\
 \hline
  1 & - & 0 & 0 & 2 \\
  2 & \textbf{1} & - & 0 & 1 \\
  3 & \textbf{1} & \textbf{1} & - & 0 \\
  4 & 0 & 1 & 2 & - \\
\end{tabular}
\begin{minipage}{0.5\textwidth}
\includegraphics[width=200px]{figures/lambda_colinear_21_31_32}
\end{minipage}
\end{frame}

% %-----------------------------------------------------------------------
% \begin{frame}
% \frametitle{$\lambda$-matrix creation for colinear sets}
% \begin{tabular}{| c || c | c | c | c |}
%     & 1 & 2 & 3 & 4 \\
%  \hline
%   1 & - & 0 & 0 & 2 \\
%   2 & 1 & - & 0 & \textbf{1} \\
%   3 & 1 & 1 & - & 0 \\
%   4 & 0 & 1 & 2 & - \\
% \end{tabular}
% \begin{minipage}{0.5\textwidth}
% \includegraphics[width=200px]{figures/lambda_colinear_24}
% \end{minipage}
% \end{frame}
% 
% %-----------------------------------------------------------------------
% \begin{frame}
% \frametitle{$\lambda$-matrix creation for colinear point sets}
% \begin{tabular}{| c || c | c | c | c |}
%     & 1 & 2 & 3 & 4 \\
%  \hline
%   1 & - & 0 & 0 & 2 \\
%   2 & 1 & - & 0 & 1 \\
%   3 & 1 & 1 & - & \textbf{0} \\
%   4 & 0 & 1 & 2 & - \\
% \end{tabular}
% \begin{minipage}{0.5\textwidth}
% \includegraphics[width=200px]{figures/lambda_colinear_34}
% \end{minipage}
% \end{frame}
% 
% %-----------------------------------------------------------------------
% \begin{frame}
% \frametitle{$\lambda$-matrix creation for colinear point sets}
% \begin{tabular}{| c || c | c | c | c |}
%     & 1 & 2 & 3 & 4 \\
%  \hline
%   1 & - & 0 & 0 & 2 \\
%   2 & 1 & - & 0 & 1 \\
%   3 & 1 & 1 & - & 0 \\
%   4 & \textbf{1} & 1 & 2 & - \\
% \end{tabular}
% \begin{minipage}{0.5\textwidth}
% \includegraphics[width=200px]{figures/lambda_colinear_41}
% \end{minipage}
% \end{frame}
% 
% %-----------------------------------------------------------------------
% \begin{frame}
% \frametitle{$\lambda$-matrix creation for colinear point sets}
% \begin{tabular}{| c || c | c | c | c |}
%     & 1 & 2 & 3 & 4 \\
%  \hline
%   1 & - & 0 & 0 & 2 \\
%   2 & 1 & - & 0 & 1 \\
%   3 & 1 & 1 & - & 0 \\
%   4 & 0 & \textbf{1} & 2 & - \\
% \end{tabular}
% \begin{minipage}{0.5\textwidth}
% \includegraphics[width=200px]{figures/lambda_colinear_42}
% \end{minipage}
% \end{frame}
% 
% %-----------------------------------------------------------------------
% \begin{frame}
% \frametitle{$\lambda$-matrix creation for colinear point sets}
% \begin{tabular}{| c || c | c | c | c |}
%     & 1 & 2 & 3 & 4 \\
%  \hline
%   1 & - & 0 & 0 & 2 \\
%   2 & 1 & - & 0 & 1 \\
%   3 & 1 & 1 & - & 0 \\
%   4 & 0 & 1 & \textbf{2} & - \\
% \end{tabular}
% \begin{minipage}{0.5\textwidth}
% \includegraphics[width=200px]{figures/lambda_colinear_43}
% \end{minipage}
% \end{frame}
% 
% %-----------------------------------------------------------------------
% \begin{frame}
% \frametitle{$\lambda$-matrix creation for colinear point sets}
% \begin{tabular}{| c || c | c | c | c |}
%     & 1 & 2 & 3 & 4 \\
%  \hline
%   1 & - & 0 & 0 & 2 \\
%   2 & 1 & - & 0 & 1 \\
%   3 & 1 & 1 & - & 0 \\
%   4 & 0 & 1 & 2 & - \\
% \end{tabular}
% \begin{minipage}{0.5\textwidth}
% \includegraphics[width=200px]{figures/lambda_colinear}
% \end{minipage}
% \end{frame}

%-----------------------------------------------------------------------
\begin{frame}
\frametitle{$\lambda$-matrix symmetry for collinear point sets}
\begin{center}
\begin{tabular}{| c || c | c | c | c |}
    & 1 & 2 & 3 & 4 \\
 \hline
  1 & - & \textcolor{red}{0} & \textcolor{violet}{0} & 2 \\
  2 & \textcolor{red}{1} & - & \textcolor{blue}{0} & 1 \\
  3 & \textcolor{violet}{1} & \textcolor{blue}{1} & - & 0 \\
  4 & 0 & 1 & 2 & - \\
\end{tabular}\\
\end{center}

one can see that this matrix is not symmetric for every point.
\begin{itemize}
\item \textcolor{red}{point 1 is collinear with 2 and 3}
\item \textcolor{violet}{point 2 is collinear with 1 and 3}
\item \textcolor{blue}{point 3 is collinear with 1 and 2}

\end{itemize}
$\rightarrow$ complete $\lambda$-matrix holds sufficient information regarding collinear points\\
$\rightarrow$ upper $\lambda$-matrix is sufficient for non collinear matrices

\end{frame}

%-----------------------------------------------------------------------
\begin{frame}
\frametitle{detection collinear candidates}
we use the presented orientation algorithm in order to detect orientation and collinearity
\begin{itemize}
\item if points are contained two collinear groups all 4 points are collinear
\begin{itemize}
  \item ABC collinear, BCD collinear $\rightarrow$ ABCD collinear
\end{itemize}
\item this also holds for more than two sets
\end{itemize}
%furthermore we found out that collinearities can be found while creating $\lambda$-matrix
$ \rightarrow $ furthermore collinearities can be found during $\lambda$-matrix creation
\end{frame}
% -----------------------------------------------------------------------
\begin{frame}
\frametitle{deterministic automatic creation}
correction method
\begin{itemize}
  \item put all points on convex hull
  \item correct single points if $\lambda$-matrix differs
  \item disadvantage:  gets very complicated, even for small amount of points
\end{itemize}
insertion method
\begin{itemize}
  \item insert points according to $\lambda$-matrix
      \begin{itemize}
       \item correct area can be found via orientation determination
      \end{itemize} 
  \item backtracking
      \begin{itemize}
       \item reverse step if area has vanished
      \end{itemize} 
  
\end{itemize}
\end{frame}
% -----------------------------------------------------------------------
\begin{frame}
\frametitle{semiautomatic approach for creating point set from $\lambda$-matrix}
\begin{itemize}
  \item automatic creation is very hard
  \item human support would make that task pretty easy
  \item first two points are inserted automatically
  \item colourise "forbidden" areas
  \item additional points are inserted according to $\lambda$-matrix
  \item moving points inside an area will not change the fingerprint
\end{itemize}
\end{frame}
% -----------------------------------------------------------------------
\begin{frame}
\frametitle{further ideas}


\begin{itemize}
  \item count empty triangles?\\no, order types are overkill for that! ($d = 2n - 2 - h$) (Eulerscher Polyedersatz)
  \item automatic creation of point set representation from $\lambda$-matrix?\\ yes, once you have this anything is possible!
  \begin{itemize} 
    \item non deterministic approach (easy)
    \item deterministic approach (hard)
    \item prolog approach (cheap! needs well-designed attributes, though)
  \end{itemize}
  
\end{itemize}
\end{frame}
% -----------------------------------------------------------------------
\begin{frame}
\frametitle{automatic creation of $\lambda$-matrices}

\begin{itemize} 
  \item idea: create all valid $\lambda$-matrices $\rightarrow$ all different point sets
  \item determine valid $\lambda$-matrices: hard problem?
  \item $\rightarrow$ our approach: generate "arbitrary" $\lambda$-matrix and check if valid by automatic drawing
\end{itemize}
  
\end{frame}
 
% -----------------------------------------------------------------------
\begin{frame}
\frametitle{goodbye}

thank you for your attention
  
\end{frame}
% -----------------------------------------------------------------------
\end{document}
